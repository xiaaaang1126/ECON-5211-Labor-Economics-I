\documentclass{article}
\usepackage[
backend=biber,
style=alphabetic,
sorting=ynt
]{biblatex}


\usepackage{geometry}
 \geometry{
 a4paper,
 total={170mm,257mm},
 left=20mm,
 top=20mm,
 }
 
\usepackage[acronym]{glossaries}
\usepackage{indentfirst}
\usepackage{biblatex}

\addbibresource{mybibliography.bib}


\makeglossaries

\newglossaryentry{entryOne}
{
        name=Glossary Entry,
        description={Glossary entries are used to provide definitions for words in your document}
}    


\title{Literature Review}
\author{Xiang Jyun, Jhang}
\date{April 9, 2024}




\begin{document}

\maketitle

\section{Main Content}

    During Pandemic period, most countries were in severe economic depress. Leading by the substantial amount of layoffs, the labor market shows a red sign of tightness follwed by a deep drop in wage for workers.

    In post-pandemic era, the recovery of workers' wage has different trend. Compared with the high-skilled worker, the low-skilled worker obtain more increment in wage, leading to the wage compression.

    I'm going to explore the wage compression trend in Taiwan, digging out the heterogeneity effect of pandemic on workers' wage.  
    

    1. Covid hurts labor market by decreasing labor demand, further reducing the wage
    2. However, in post-Covid era, low-skilled worker actually recover their wag back more than high-skilled worker, leading to a phenomenon called wage compression
    3. Is wage compression prevailing in Taiwan as well? That is, do low-skilled workers' wage increase more than high-skilled worker?


    
\medskip



\printbibliography

\end{document}
